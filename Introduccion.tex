% Chapter Template

\chapter{Introducción} % Main chapter title

\label{Chapter1} % Change X to a consecutive number; for referencing this chapter elsewhere, use \ref{ChapterX}

%----------------------------------------------------------------------------------------
%	SECTION 1
%----------------------------------------------------------------------------------------

\section{Seguridad en Internet}

Es posible encontrar múltiples definiciones de lo que se entiende por seguridad en la red o Internet. Por ejemplo, Stallings se refiere a ello como la protección que se proporciona a un sistema de la información para preservar la integridad, disponibilidad y confidencialidad de sus recursos, tanto software como hardware \cite{Stallings2016}. Por otro lado, la empresa Cisco la define como la actividad destinada a proteger la usabilidad y la integridad de la red y datos. Al igual que la anterior, engloba medios software y hardware \cite{cisco}. Estas son tan solo dos ejemplos de las muchas acepciones que existen, pero es posible observar que coinciden en gran medida en los aspectos que la seguridad, en términos de Internet, debería garantizar. Ambas también establecen los mismos objetivos a proteger: medios hardware y software. Es importante entender en que consisten la integridad, la disponibilidad y la confidencialidad, conjunto conocido como \textit{CIA} (\textit{Confidenciality, Integrity and Availability}). La confidencialidad asegura que datos de carácter privado no sean accedidos por personas no autorizadas; la disponibilidad permite que datos o cualquier otro tipo de recurso pueda ser utilizado sin ningún tipo de impedimento y, finalmente, la integridad preserva el contenido de los datos o comportamiento de un sistema, de manera que estos no sean modificados por alguien desautorizado. Todos estos términos pueden aplicarse a un sistema aislado, pero cuando este sistema pasa a estar en una red de millones de nodos, las amenazas se múltiplican.  

%-----------------------------------
%	SUBSECTION 1
%-----------------------------------
\subsection{Tipos de atacantes}
Se distinguen distintos tipos de atacantes: hackers, criminales y empleados. Los hackers suelen realizar ataques buscando la emoción de conseguir acceder a un sistema restringido y el reconocimiento del resto de la comunidad hacker. Son frecuentes los ataques de oportunidad que aprovechan alguna vulnerabilidad para acceder a información que luego comparten en la red. Los segundos atacantes, los criminales, constituyen bandas de hackers que se asocian para llevar a cabo ataques con fines lucrativos, generalmente contra servicios de comercio electrónico. Tratan de hacerse con datos bancarios y tarjetas de crédito que después utilizan a expensas de la víctima o venden en la red.
Estos grupos, que se han expandido por toda la red, suponen una amenaza común para todos los sistemas basados en Internet, buscan objetivos concretos y, en ocasiones, son contratados por gobiernos u otras organizaciones. Por úlimo, los empleados son individuos que ya se encuentran dentro del sistema y conocen su estructura. Sus ataques pueden estar motivados por venganza contra la organización en la que trabajan o sencillamente, por un sentimiento de derecho. Resultan, por lo tanto, los ataques más difíciles de detectar y prevenir, y solamente políticas de acceso y monitorización dentro de la organización ayudan a evitarlos\cite{Stallings2016}.

%-----------------------------------
%	SUBSECTION 2
%-----------------------------------
\subsection{Ataques de seguridad y sus motivaciones}
Las normas \textit{X.800} y \textit {RFC 4949} clasifican los ataques en dos categorías: pasivos y activos. Los ataques pasivos serían aquellos que extraen información de un sistema, pero no alteran en modo alguno a sus recursos. Un ejemplo sería la monitorización de las transmisiones realizadas entre dos sistemas, accediendo a esta información. Por otro lado, los ataques activos sí que afectan los a los recursos de un sistema e incluso a su funcionamiento. Dentro de este tipo de ataques se encuentran la suplantación, cuando un individuo u organización finge ser otra distinta; reenvio de información capturada previamente sin autorización; modificación de mensajes o la denegación de servicio, que impide el acceso normal a un servicio\cite{Stallings2016}.
En lo que respecta a las motivaciones, la gran mayoría de ataques están conducidos por el espionaje o un interés financiero. Otras motivaciones serían la diversión, el resentimiento o la ideología. No obstante, hay que tener en cuanto que muchos casos de extorsión no son reportados y confirmados, por lo que las cifras recogidas en las estadísticas no reflejan la totalidad de los ataques. Aún así, es posible contar con una referencia de los fines que persiguen algunos conocidos ataques\cite{DBIR2017}: 
\begin{itemize}
	\item Financieros: uso de credenciales robadas, uso de backdoor, spyware, phising, malware para exportar data, c2.
	\item Espionaje: phising, c2, uso de backdoor.
	\item Resto: abuso de privilegios
\end{itemize}




\subsection{Algunas cifras concretas}
En relación a lo anterior el DBIR (\textit{Data Breach Investigations Report}) de 2017, un informe realizado por Verizon en el que participan 65 organizaciones, analiza el estado de la ciberseguridad.Según este informe hubo 1616 ataques durante el año 2016, de los que 828 supusieron la revelación de datos confidenciales. Este informe también proporciona los tipos de ataques más conocidos, así como los actores que los perpetran y sus motivaciones. Plasmando en cifras lo referido en la anterior sección, el 66\% de los ataques tenían una motivación financiera y el 33\% de espionaje. Menos del 1\% de los ataques fueron motivados por ideología o diversión. Además, el 99\% de estos ataques los llevaron a cabo individuos u organizaciones externas. Otro dato interesantes que recoge el informe es que la táctica más empleada es la de phising y que la mayoría de estos ataques van seguidos por la instación de algún tipo de malware. Finalmente, cabe mencionar que el 81\% de las brechas de seguridad se produjeron debido a credenciales inseguras o robadas. Este informe además analizar las estadísticas de los ataques reportados, proporciona algunos consejos en base a los resultados para tratar de evitarlos. Destaca, sobretodo, la necesidad de concienciar y educar acerca de las amenzadas y riesgos que existen. También pone el foco en la importancia que supone la detección temprana de un ataque y la localización de la fuente del ataque\cite{DBIR2017}. En definitiva, se trata tan solo de un informe, pero pone de manifiesto la magnitud del problema que representa la seguridad en internet.




%----------------------------------------------------------------------------------------
%	SECTION 2
%----------------------------------------------------------------------------------------

\section{Sección 2}

Sed ullamcorper quam eu nisl interdum at interdum enim egestas. Aliquam placerat justo sed lectus lobortis ut porta nisl porttitor. Vestibulum mi dolor, lacinia molestie gravida at, tempus vitae ligula. Donec eget quam sapien, in viverra eros. Donec pellentesque justo a massa fringilla non vestibulum metus vestibulum. Vestibulum in orci quis felis tempor lacinia. Vivamus ornare ultrices facilisis. Ut hendrerit volutpat vulputate. Morbi condimentum venenatis augue, id porta ipsum vulputate in. Curabitur luctus tempus justo. Vestibulum risus lectus, adipiscing nec condimentum quis, condimentum nec nisl. Aliquam dictum sagittis velit sed iaculis. Morbi tristique augue sit amet nulla pulvinar id facilisis ligula mollis. Nam elit libero, tincidunt ut aliquam at, molestie in quam. Aenean rhoncus vehicula hendrerit.