% Chapter Template

\chapter{Introducción} % Main chapter title

\label{Chapter1} % Change X to a consecutive number; for referencing this chapter elsewhere, use \ref{ChapterX}

%----------------------------------------------------------------------------------------
%	SECTION 1
%----------------------------------------------------------------------------------------

\section{Seguridad en Internet}

Es posible encontrar múltiples definiciones de lo que se entiende por seguridad en la red o Internet. Por ejemplo, Stallings se refiere a ello como la protección que se proporciona a un sistema de la información para preservar la integridad, disponibilidad y confidencialidad de sus recursos, tanto software como hardware \cite{Stallings2016}. Por otro lado, la empresa Cisco la define como la actividad destinada a proteger la usabilidad y la integridad de la red y datos. Al igual que la anterior, engloba medios software y hardware \cite{cisco}. Estas son tan solo dos ejemplos de las muchas acepciones que existen, pero es posible observar que coinciden en gran medida en los aspectos que la seguridad, en términos de Internet, debería garantizar. Ambas también establecen los mismos objetivos a proteger: medios hardware y software. Es importante entender en que consisten la integridad, la disponibilidad y la confidencialidad, conjunto conocido como \textit{CIA} (\textit{Confidenciality, Integrity and Availability}). La confidencialidad asegura que datos de carácter privado no sean accedidos por personas no autorizadas; la disponibilidad permite que datos o cualquier otro tipo de recurso pueda ser utilizado sin ningún tipo de impedimento y, finalmente, la integridad preserva el contenido de los datos o comportamiento de un sistema, de manera que estos no sean modificados por alguien desautorizado. Todos estos términos pueden aplicarse a un sistema aislado, pero cuando este sistema pasa a estar en una red de millones de nodos, las amenazas se múltiplican.  

%-----------------------------------
%	SUBSECTION 1
%-----------------------------------
\subsection{Tipos de atacantes}
Se distinguen distintos tipos de atacantes: hackers, criminales y empleados. Los hackers suelen realizar ataques buscando la emoción de conseguir acceder a un sistema restringido y el reconocimiento del resto de la comunidad hacker. Son frecuentes los ataques de oportunidad que aprovechan alguna vulnerabilidad para acceder a información que luego comparten en la red. Los segundos atacantes, los criminales, constituyen bandas de hackers que se asocian para llevar a cabo ataques con fines lucrativos, generalmente contra servicios de comercio electrónico. Tratan de hacerse con datos bancarios y tarjetas de crédito que después utilizan a expensas de la víctima o venden en la red.
Estos grupos, que se han expandido por toda la red, suponen una amenaza común para todos los sistemas basados en Internet, buscan objetivos concretos y, en ocasiones, son contratados por gobiernos u otras organizaciones. Por úlimo, los empleados son individuos que ya se encuentran dentro del sistema y conocen su estructura. Sus ataques pueden estar motivados por venganza contra la organización en la que trabajan o sencillamente, por un sentimiento de derecho. Resultan, por lo tanto, los ataques más difíciles de detectar y prevenir, y solamente políticas de acceso y monitorización dentro de la organización ayudan a evitarlos\cite{Stallings2016}.

%-----------------------------------
%	SUBSECTION 2
%-----------------------------------
\subsection{Ataques de seguridad y sus motivaciones}
Las normas \textit{X.800} y \textit {RFC 4949} clasifican los ataques en dos categorías: pasivos y activos. Los ataques pasivos serían aquellos que extraen información de un sistema, pero no alteran en modo alguno a sus recursos. Un ejemplo sería la monitorización de las transmisiones realizadas entre dos sistemas, accediendo a esta información. Por otro lado, los ataques activos sí que afectan a los recursos de un sistema e incluso a su funcionamiento. Dentro de este tipo de ataques se encuentran la suplantación, cuando un individuo u organización finge ser otra distinta; reenvio de información capturada previamente sin autorización; modificación de mensajes o la denegación de servicio, que impide el acceso normal a un servicio\cite{Stallings2016}.
En lo que respecta a las motivaciones, la gran mayoría de ataques están conducidos por el espionaje o un interés financiero. Otras motivaciones serían la diversión, el resentimiento o la ideología. No obstante, hay que tener en cuanto que muchos casos de extorsión no son reportados y confirmados, por lo que las cifras recogidas en las estadísticas no reflejan la totalidad de los ataques. Aún así, es posible contar con una referencia de los fines que persiguen algunos conocidos ataques\cite{DBIR2017}: 
\begin{itemize}
	\item Financieros: uso de credenciales robadas, uso de backdoor, spyware, phising, malware para exportar data, c2.
	\item Espionaje: phising, c2, uso de backdoor.
	\item Resto: abuso de privilegios
\end{itemize}


\section{Algunas cifras concretas}
En relación a lo anterior, existen multitud de estudios e informe que tratan de recabar información acerca del estado de la seguridad en Internet partiendo de diversas fuentes, como encuestas o ataques sufridos. Pese a que gran parte de estos estudios están realizados por empresas privadas, resultan útiles para obtener una perspectiva global del problema que supone la seguridad en Internet.


\subsection{DBIR 2017}
El DBIR (\textit{Data Breach Investigations Report}), un informe realizado por Verizon en el que participan 65 organizaciones, analiza el estado de la ciberseguridad.Según este informe hubo 1616 ataques durante el año 2016, de los que 828 supusieron la revelación de datos confidenciales. Este informe también proporciona los tipos de ataques más conocidos, así como los actores que los perpetran y sus motivaciones. Plasmando en cifras lo referido en la anterior sección, el 66\% de los ataques tenían una motivación financiera y el 33\% de espionaje. Menos del 1\% de los ataques fueron motivados por ideología o diversión. Además, el 99\% de estos ataques los llevaron a cabo individuos u organizaciones externas. Otro dato interesantes que recoge el informe es que la táctica más empleada es la de phising y que la mayoría de estos ataques van seguidos por la instación de algún tipo de malware. Finalmente, cabe mencionar que el 81\% de las brechas de seguridad se produjeron debido a credenciales inseguras o robadas. Este informe además de analizar las estadísticas de los ataques reportados, proporciona algunos consejos en base a los resultados para tratar de evitarlos. Destaca, sobretodo, la necesidad de concienciar y educar acerca de las amenzadas y riesgos que existen. También pone el foco en la importancia que supone la detección temprana de un ataque y la localización de la fuente del ataque\cite{DBIR2017}.
\begin{figure}[t]
\centering
\includegraphics[width=1\textwidth]{images/userBehavior.png}
\caption{Identificación de patrones de comportamiento de usario normal}
\label{fig:behavior}
\end{figure}

\subsection{Cisco 2017 Annual Cybersecurity Report}
El grupo de investigación de seguridad de Cisco publica cada año este informe, para ayudar a las organizaciones a hacer frente a las amenazas y riesgos que surgen constantemente en la red. Entre los datos recogidos por el informe cabe destacar las razones que impiden la adopción de sistemas u otras medidas de seguridad en muchas empresas. El 35\% carecía de presupuesto, para el 28\% presentaba problemas de compatibilidad, el 25\% por la certificación y el 25 \% restante por falta de talento. Por este motivo, apenas la mitad de las alertas de seguridad que se reciben son investigadas. Cabe mencionar también el hecho de que aquellas organizaciones que aún no han sufrido ninguna brecha de seguridad están convencidas de que su red es segura, aunque esta seguridad parece cuestionable si se tiene en cuenta el grado de afectación que supone para cualquier empresa que su sistema se vea comprometido. Casi un cuarto de las empresas perdió alguna oportunidad de negocio al sufrir un ataque y 1 de cada 5 perdió clientes. Este estudio también muestra que muchas de las empresas recurren a las soluciones de seguridad de varias empresas especializadas, por lo general más de 5, con varios productos distintos también. Todo ello supone una complejidad extra que dificulta la automatización de tareas, algo fundamental a la hora de mejorar la seguridad de un sistema. Por ejemplo, distinguir un comportamiento anómalo y sospechoso del que, según los patrones, resulta normal requiere un proceso de varias etapas que solo puede lograrse con automatización\ref{fig:behavior}.\\
En lo que se refiere a los ataques, los datos revelan que en la mayoría de ellos se distinguen las siguientes fases:
\begin{itemize}
	\item Reconomiento: los atacantes investigan, identifican y seleccionan a sus víctimas.
	\item Armamento: generación de paquetes con malware que permite el acceso remoto aprovechando una vulnerabilidad.
	\item Distribución: la carga anterior se hace llegar mediante correo, ficheros adjuntos, etc.
	\item Instalación: en el objetivo, el malware genera una puerta trasera que permite el acceso permanente de los atacantes.
\end{itemize}
Frente a los ataques, una de las medidas que propone cisco para conocer el progreso de las medidas de seguridad es el TTD (\textit{Time To Detec}. Lo define como el intervalo de tiempo que transcurre desde que un sistema se ve comprometido hasta que la amenaza es detectada. Las amenazas y ataques evolucionan muy rápidamente y en ocasiones resulta difícil identificar un ataque, aunque este sea conocido en la comunidad. De la misma manera que los sistemas de seguridad trabajan en mejorar el TTD, los atacantes desarrollan nuevas técnicas y estrategias para evitar ser detectados y disponer así de más tiempo  para perpetrar su ataque. Esta mejora en los ataques se puede medir con el TTE (\textit{Time To Evolve}, el tiempo que tarda un atacante en modificar el modo en que cierto malware es distribuido o en cambiar de táctica. El hecho de que los ataques evolucionen con tanta rapidez denota, a su vez, las mejoras que experimentan los sistemas de seguridad.\\
Este progreso constante en ambos partes, ha supuesto un incremento en el personal dedicado exclusivamente a la seguridad en las empresas. Frente a las 25 personas que se registraron de media en cada organización durante el año 2015, el año 2016 esta cifra era de 33. El interés por combatir el progreso de las amenazas en Internet radica en el impacto que estos ataques tienen en una organización o empresa. La repercusión no se limita a cortes de servicio, con la consecuente pérdida de dinero, sino que afecta gravemente a la reputación. De hecho, el 33\% de las organizaciones encuestadas tuvieron que hacer frente a la publicación involuntaria de ataques que sufrieron.\\
El informe concluye que toda organización es susceptible de sufrir un ciberataque, siendo necesaria una constante mejora en los medios de seguridad teniendo en cuenta, además, las limitaciones de presupuesto y compatibilidad, entre otras\cite{ciscoReport}

\section{Búsqueda de soluciones contra los ciberataques}
Los informes comentados anteriormente reflejan tan solo una pequeña parte del total de informes elaborados por empresas y organizaciones. Pese a que ambos han sido llevados a cabo por empresas privadas con sus propios interes y es necesario estudiarlos con una actitud crítica, sí que ponen de manifiesto la magnitud del problema que supone la seguridad en Internet. Como puede extraerse de los datos, en la red toda organización es susceptible de convertirse en víctima de un ataque y la necesidad de sistemas seguros se mantiene.\\ 
Se ha discutido anteriormente cómo son y en qué consisten algunos de los ataques más frecuentes de la red, ¿pero qué soluciones existen para hacerlos frente? Se han desarrollado multitud de estándars que abarcan desde técnicas de gestión de ataques, hasta arquitecturas recomendables. Algunas de las instituciones más importantes que realizan esta tarea son el NIST(\textit{National Institue of Standards and Technology}), una agencia federal de Estados Unidos; o la ISOC (\textit{Internet Society}), que elabora RFCs(\textit{Requests For Comments})\cite{Stallings2016}. Además de estos estándares, hay diversos productos concretos que los expertos en ciberseguridad utilizan. La organización secTools proporciona una lista de las herramientas más populares\cite{secTools}
\begin{itemize}
	\item \textbf{Wireshark}: se trata de un analizador de protocolos, que permite estudiar el tráfico de red.
	\item \textbf{Metasploit}: herramienta para probar y desarrollar código de \textit{exploits}.
	\item \textbf{Nessus}: permite realizar escáneres.
	\item \textbf{Aircrack}: utilidad para romper algoritmos de encriptación y recuperar contraseñas para los protocolos 802.11a/b/g WEP and WPA.
	\item \textbf{Snort}: es un IDS(\textit{Intrusion Detection Sistem}), que tal y como su nombre indica, se emplea para detectar instrusiones.
	\item \textbf{Cain and Abel}: herramienta para recuperar contraseñas en Windows.
	\item \textbf{BackTrack}: distribución de Linux que engloba herramientas forenses y de seguridad.
	\item \textbf{Netcat}: permite leer y escribir datos a través de conexiones TCP y UTP.
	\item \textbf{Tcpdump}: al igual que Wireshark analiza el tráfico de red, pero se trata de una herramienta de línea de comandos.
	\item \textbf{John the Ripper}: se utiliza para \textit{crackear} contraseñas en sistemas Unix y Mac, permitiendo detectar contraseñas débiles.
\end{itemize}
Como puede comprobarse, gran parte de estas herramientas son empleadas tanto para auditorías, como para perpetrar ataques, puesto que a la hora de comprobar la eficacia de cualquier sistema de seguridad, resulta fundamental ponerlo a prueba con ataques reales. Y es ahí donde entran en juego lo que se conoce como \textit{honeypots}. 
\subsection{Honeypots}
\begin{figure}[t]
\centering
\includegraphics[width=0.8\textwidth]{images/clienthoneypot.png}
\caption{Funcionamiento de un cliente \textit{honeypot}}
\label{fig:clienthoneypot}
\end{figure}
Los \textit{honeypots} son sistemas aislados en la red que actúan como señuelos de cara a posibles atacantes, evitando así que accedan a información crítica. Por este motivo, han de simular eficazmente un sistema productivo, pero carente de datos reales o sensibles, de manera que el atacante no sea consciente del engaño y tenga una falsa sensación de seguridad. Un \textit{honeypot} está diseñado con el objetivo de retener al atacante el mayor tiempo posible, siendo éste uno de sus mayores atractivos, pues permite estudiar su comportamiento. Tal y como se ha mencionado anteriormente, resulta de vital importancia entender cómo piensa y actúa un \textit{hacker} para diseñar mejores estrategias de defensa\cite{Stallings2016}. Existen diferentes tipos de \textit{honeypots}. Si se atiende a su funcionamiento y ámbito de actuación, tradicionalmente han actuado de forma pasiva en el lado del servidor, esperando intrusiones y monitorizando los ataques que se produjeran. Un ejemplo de este tipo sería \textit{Kippo}, un SSH  \textit{honeypot}, que permite levantar puertos que esperen conexiones SSH de manera que los atacantes puedan acceder al sistema e interactuar con un sistema de ficheros ficticios, mientras toda esta actividad queda registrada\cite{THW}. Frente a este modelo pasivo se encuentran los clientes \textit{honeypot}, que se encuentran en el lado del cliente e investigan posibles ataques desde servidores de manera activa. Cuando se detecta una posible amenaza o intrusión, el sistema realiza una serie de peticiones para detectar servidores malignos, tal y como se ilustra en la figura \ref{fig:clienthoneypot} \cite{HoneynetProject}.     

Dentro de este tipo de \textit{honeypots} se encuentra \textit{Shelia}, que emula el comportamiento de un usuario que accediese a los enlaces que incluyen los correos de spam, detectando así qué servidores suponen un peligro para la seguridad del sistema\cite{shelia}.  

\subsection{La evolución hacia las honeynets}
Partiendo del concepto de \textit{honeypot} y teniendo presente la necesidad de desarrollar sistemas económicos que confronten las limitaciones presupuestarias que existen en el ámbito de la ciberseguridad, el presente proyecto tiene por objetivo desarrollar una \textit{honeynet}, que emplee máquinas virtuales creadas dinámicamente para gestionar ataques e intrusiones. Para ello, aunará la capacidad de detección de un IDS con herramientas de virtualización, de manera que el resultado sea un sistema integrado. 
